\documentclass[fleqn,10pt]{wlpeerj}
%% DE created stub file June 30, 2014 initially set up for PeerJ
%% Have at - change whatever you like.
%% Text is being tracked using Mercurial for revision control.  

% some packages here 
\usepackage{graphicx}
\usepackage{siunitx}
\usepackage[hidelinks]{hyperref}
\usepackage{lineno} % for review only

\title{Fun things you can do with a bunch of GoPros}

\author[1,2]{Brandon Jackson \thanks{author for correspondence: brandon.e.jackson@gmail.com}}
\author[2]{Dennis Evangelista}
\author[2]{Ty Hedrick}
\affil[1]{Longwood College, Charlottesville, VA}
\affil[2]{University of North Carolina at Chapel Hill, NC 27599-3280, USA}

\keywords{videography, photogrammetry, kinematics, multiple cameras, calibration}

\begin{abstract}
Ecological, behavioral, and biomechanical studies often need to quantify movement and behavior in three dimensions.  In laboratory studies, a major tool to accomplish these is the use of multiple, calibrated high-speed cameras.  Until very recently, complexity, weight and cost of such cameras has made their deployment in field situations risky; furthermore such cameras are often not affordable for early career researchers, teaching use by undergraduates, or for those not in biomechanics who don't have an overriding primary need for such toys.  Here we describe a solution and tool set using multiple inexpensive cameras to bridge such needs.  The availability of lower cost, portable and rugged solutions (and the likely future ubiquity of inexpensive cameras of reasonable performance) has promise to open up new areas of biological study by providing precise, 3D tracking and quantification of movement to more researchers. 
\end{abstract}

\begin{document}

\flushbottom
\maketitle
\thispagestyle{empty}

%% for line numbers
%\setpagewiselinenumbers
\modulolinenumbers[5]
\linenumbers

\section*{Introduction}

Introduction here. Citations here \citep{Theriault:2014, Bradski:2004}. 

3D reconstruction of position is an important technique that can be used to quantitatively study (STUFF).  For example (STUFF, Citations). As cameras become more capable, lightweight, and affordable, such techniques are increasingly attractive for use in teaching or in field situations.  

Calibration of multiple cameras and using them to reconstruct position can be nontrival.  To use multiple cameras to reconstruct 3D positions requires knowledge of the cameras' intrinsic parameters, such as focal length, principal point, and distortion coefficients.  It also requires knowledge of the extrinsic parameters: the relative positions and orientations of the cameras with respect to one another. Finally, 3D reconstruction generally requires knowing that the frames are synchronized in time.  While a number of tools exist to actually perform the calibration (Borguet, Theriault et al) they can still be daunting to general users.

In this paper, we provide a simple work flow and tools aimed specifically at low cost 3D reconstruction using multiple consumer-grade cameras (specifically, the GoPro Hero 3 series; although we have used the techniques here with Flip MinoHD and various models of digital SLR cameras).  To ease adoption of such techniques, we provide a database of calibrations; simple tools for obtaining the intrinsics from printed patterns; and methods for obtaining audio synchronization, wand tracking (and keypoint detection?) used to obtain extrinsics in the field. Our goal in this paper is simplify the employment of 3D reconstruction techniques so that they may be used in ecological field studies, undergraduate teaching of biomechanics, etc etc etc. 

\subsection*{How does 3D reconstruction work in a nutshell?}
\subsection*{Ecologist-proof roadmap of how to set up to do it}

\section*{Methods and materials}
\subsection*{Cameras}
\subsection*{Software tools}
\subsection*{Laboratory calibration of camera intrinsics}
\subsection*{Field deployment and field calibration of camera extrinsics}
\subsection*{3D reconstruction}

\section*{Results and discussion}
\subsection*{Simple lab example}
\subsection*{Simple field example}

\section*{Acknowledgements}
Thank you everyone.

\bibliography{gopro}
\end{document}